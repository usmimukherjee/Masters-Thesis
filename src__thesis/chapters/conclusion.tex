\chapter{Conclusion and Future Work}

\section{Conclusion}
Missing information in bug reports makes bug resolution challenging for developers~\cite{zou2018practitioners}. Bug reports often do not contain all the required information for reproducing or resolving the bug~\cite{zhang2017bug}. Developers pose follow-up questions to bug reporters seeking missing information. However, bug reporters often fail to provide answers in a timely fashion. Furthermore, complex contextual information in bug reports could make bug understanding challenging. Newcomers or novice developers of a project might thus need additional help to understand or resolve a bug accurately. First, there have been several existing studies that provide complementary information to bug reports using automated techniques~\cite{imran2021automatically, tian2017apibot,bansal2021neural, lu2021beat,breu2010information}. However, there has been little research investigating the follow-up questions from bug reports or their answers. Second, existing studies offer complementary information to support bug understanding, leveraging external resources and past relevant bug reports~\cite{correa2013samekana,zhang2017bug,dit2008improving,moran2018enhancing,fazzini2022enhancing,xu2017answerbot}. However, they do not focus on the domain-specific terms or jargon, which warrants for further investigation. This thesis addresses the issue of missing information by complementing deficient bug reports with additional information. Thus, we conduct two separate but complementary studies (Chapter~\ref{chapter:BugMentor}) and Chapter~\ref{Chap2:BugEnricher}), and we have the following outcomes. \par

\begin{itemize}
    \item The first study (Chapter~\ref{chapter:BugMentor}) proposes a novel technique --- BugMentor --- that can offer relevant answers to follow-up questions from bug reports by combining structured information retrieval and neural text generation. We evaluate our technique on top 20 (5 Java, 5 Python, 5 C++ and 5 JavaScript) GitHub projects and four evaluation metrics (i.e., BLEU, Semantic Similarity, WMD and METEOR). Our evaluation using four performance metrics shows that BugMentor can generate understandable and good answers to follow-up questions, as per Google's Standard. Our technique was also able to outperform three existing baselines across all four metrics. We also evaluate BugMentor using a user study using 10 developers. The developers found the answers from BugMentor to be more accurate, precise, concise and useful compared to the baseline answers. Thus, BugMentor has the potential to support bug resolution with complementary information in the form of answers to follow-up questions. 

    \item The second study (Chapter~\ref{Chap2:BugEnricher}) proposes a novel technique --- BugEnricher --- that generates explanations to software-specific terms by learning from thousands of domain-specific terms and their explanations from Stack Overflow, official API documentation, and an online glossary. We evaluate our technique on Python, Java, and Miscellaneous (a.k.a., cross-language) and three evaluation metrics (i.e., BLEU, Semantic Similarity, and METEOR). BugEnricher is able to generate understandable to good explanations to the domain-specific terms when compared against the ground truth as per Google's standards. Our technique was also able to outperform two existing baselines across three metrics. Furthermore, we also conduct a case study using duplicate bug reports and attempt to enrich duplicate bug reports that are textually dissimilar. We find that the enrichment of bug reports by BugEnricher improved the performance of an existing technique for duplicate bug detection.
\end{itemize}

\section{Future Work}
We have several directions for future research from both our studies. We present the potential future work for each study below.

\subsection{BugMentor}
There are several avenues for future work from BugMentor. First, we plan to design a tool that can be integrated with real-world platforms like GitHub or JIRA to assist the bug reporters and the developers in their work. In particular, real-time feedback from the stakeholders can be leveraged to improve our retrieval algorithm.
Second, BugMentor does not use a fine-tuned version of CodeT5. If the labelled dataset (Refer to Section~\ref{sec:groundtruth} for details) can be extended, it can be used to fine-tune the CodeT5 model, which could lead to better answers.

\subsection{BugEnricher}
In the future, there are numerous potential directions for BugEnircher. First, we plan to investigate how image and video data attached to the bug report can provide additional context to the bug report. Second, we plan on fine-tuning BugEnricher with vocabulary from popular libraries from both Python and Java and improving the explanations further to generate code examples and investigate if these enriched bug reports can improve existing automated bug localization techniques.
 
